\chapter{Execution Environment}

The Tuim's \texttt{run} command is the command used to pass a program
to the Interpreter, then the Interpreter prepare all for execution and
the \texttt{run} command start the execution of the program.

The correctness of \texttt{run} and Interpreter's
work is constrained to the correct
environment configuration.
The environment variable \texttt{TUIM\_HOME}
shall holds the directory path where Tuim is installed.
When Tuim is used to run a program named \texttt{prog} then it find
for the executable placed at \texttt{bin/prog.elf} inside \texttt{TUIM\_HOME}.
If the machine that must run the program is the host machine
then the Interpreter find for libraries inside the subdirectory
\texttt{lib} of \texttt{TUIM\_HOME},
otherwise, if is a machine named \texttt{m} that must run the program then
the Tuim first find for another implementation of the Interpreter
in the library \texttt{lib/libtuim-m.so} and that Interpreter find for libraries
inside \textt{lib/m} subdirectory.

\section{Executing from command line}

\subsubsection*{SYNOPSIS}

\begin{lstlisting}[style=bash]
tuim run [-m machine] <program|executable> [arguments...]
\end{lstlisting}

\subsubsection*{DESCRIPTION}

Execute the Tuim's \texttt{run} command,
optionally using \texttt{machine} as machine,
using \texttt{program} as program's name,
if such name contains slashes
it is interpreted as the a \texttt{executable} path.

\section{Executing from C code}

\subsubsection*{SYNOPSIS}

\begin{lstlisting}[style=c]
#include <tuim.h>

int tuim_run(const char *machine, char **argv);
\end{lstlisting}

\subsubsection*{DESCRIPTION}

The function \texttt{tuim\_run} execute the Tuim's \texttt{run} command.
If the pointer \texttt{machine} is non-null it
uses the null-terminated string pointed to by
\texttt{machine} as machine's name.
It uses the null-terminated string pointed to by
\texttt{argv[0]} as program's name,
if such string contains slashes
it is interpreted as the a executable path.
