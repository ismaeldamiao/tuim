%   Copyright (c) 2024 I.F.F. dos SANTOS <ismaellxd@gmail.com>

%   Permission is hereby granted, free of charge, to any person obtaining a copy 
%   of this software and associated documentation files (the “Software”), to 
%   deal in the Software without restriction, including without limitation the 
%   rights to use, copy, modify, merge, publish, distribute, sublicense, and/or 
%   sell copies of the Software, and to permit persons to whom the Software is 
%   furnished to do so, subject to the following conditions:

%   The above copyright notice and this permission notice shall be included in 
%   all copies or substantial portions of the Software.

%   THE SOFTWARE IS PROVIDED “AS IS”, WITHOUT WARRANTY OF ANY KIND, EXPRESS OR 
%   IMPLIED, INCLUDING BUT NOT LIMITED TO THE WARRANTIES OF MERCHANTABILITY, 
%   FITNESS FOR A PARTICULAR PURPOSE AND NONINFRINGEMENT. IN NO EVENT SHALL THE 
%   AUTHORS OR COPYRIGHT HOLDERS BE LIABLE FOR ANY CLAIM, DAMAGES OR OTHER 
%   LIABILITY, WHETHER IN AN ACTION OF CONTRACT, TORT OR OTHERWISE, ARISING 
%   FROM, OUT OF OR IN CONNECTION WITH THE SOFTWARE OR THE USE OR OTHER DEALINGS 
%   IN THE SOFTWARE.

\documentclass[
   % -- opções da classe memoir --
   article,                      % indica que é um artigo acadêmico
   10pt,                         % tamanho da fonte.
   openright,                    % capítulos começam em pág ímpar (insere página vazia caso preciso).
   oneside,                      % para impressão em recto. Oposto a toweside.
   a4paper,                      % tamanho do papel. 
   sumario = tradicional,        % Para indentar os sumários
   % -- opções do pacote babel --
   english,                      % idioma adicional para hifenização.
   xcolor=table                  % Tabelas coloridadas.
]{abntex2}
% --------------------------------------
% PACOTES
% --------------------------------------
%  Encoding: UTF-8
%
% %%%%%%%%%%%%%%%%%%%%%%%%%%%%%%%%%%%%%%%%%%%%%%%%%%%%%%%%%%%%%%%%%%%%%%%%%%%% %
%  Copyright (C) 2023 I. F. F. dos SANTOS
%
%  Permission is hereby granted, free of charge, to any person obtaining a copy 
%  of this software and associated documentation files (the “Software”), to 
%  deal in the Software without restriction, including without limitation the 
%  rights to use, copy, modify, merge, publish, distribute, sublicense, and/or
%  sell copies of the Software, and to permit persons to whom the Software is 
%  furnished to do so, subject to the following conditions:
%
%  The above copyright notice and this permission notice shall be included in 
%  all copies or substantial portions of the Software.
%
%  THE SOFTWARE IS PROVIDED “AS IS”, WITHOUT WARRANTY OF ANY KIND, EXPRESS OR 
%  IMPLIED, INCLUDING BUT NOT LIMITED TO THE WARRANTIES OF MERCHANTABILITY, 
%  FITNESS FOR A PARTICULAR PURPOSE AND NONINFRINGEMENT. IN NO EVENT SHALL THE 
%  AUTHORS OR COPYRIGHT HOLDERS BE LIABLE FOR ANY CLAIM, DAMAGES OR OTHER 
%  LIABILITY, WHETHER IN AN ACTION OF CONTRACT, TORT OR OTHERWISE, ARISING 
%  FROM, OUT OF OR IN CONNECTION WITH THE SOFTWARE OR THE USE OR OTHER DEALINGS 
%  IN THE SOFTWARE.
% %%%%%%%%%%%%%%%%%%%%%%%%%%%%%%%%%%%%%%%%%%%%%%%%%%%%%%%%%%%%%%%%%%%%%%%%%%%% %

\renewcommand{\ABNTEXchapterfont}{\bfseries\rmfamily}
\renewcommand{\ABNTEXsectionfont}{\bfseries\rmfamily}
\renewcommand{\ABNTEXsubsectionfont}{\bfseries\rmfamily}
\renewcommand{\ABNTEXsubsubsectionfont}{\bfseries\rmfamily}

\renewcommand{\ABNTEXchapterfontsize}{\huge}
\renewcommand{\ABNTEXsectionfontsize}{\LARGE}
\renewcommand{\ABNTEXsubsectionfontsize}{\large}
\renewcommand{\ABNTEXsubsubsectionfontsize}{\normalsize}

\tituloestrangeiro{\vspace{-2em}}
\setlength{\ABNTEXsignwidth}{12cm}

\renewenvironment{dedicatoria}{}{}
\renewenvironment{agradecimentos}{}{}
\renewenvironment{epigrafe}{}{}

\providecommand{\imprimirdedicatoria}{}
\providecommand{\imprimiragradecimentos}{}
\providecommand{\imprimirepigrafe}{}

\renewcommand{\imprimirdata}{\today}

\renewcommand{\dedicatoria}[1]{\renewcommand{\imprimirdedicatoria}{#1}}
\renewcommand{\agradecimentos}[1]{\renewcommand{\imprimiragradecimentos}{#1}}
\renewcommand{\epigrafe}[1]{\renewcommand{\imprimirepigrafe}{#1}}


\makeatletter
\providecommand{\pretextualpreconfigurado}{
   \abntex@ifnotempty{\imprimirdedicatoria}{
      \PRIVATEbookmarkthis{\dedicatorianame}
      \vspace*{\fill}%
      \hspace*{.45\textwidth}
      \begin{minipage}{.5\textwidth}
         \imprimirdedicatoria
      \end{minipage}
      \PRIVATEclearpageifneeded
   }
   \abntex@ifnotempty{\imprimiragradecimentos}{
      \pretextualchapter{\agradecimentosname}
      \imprimiragradecimentos
      \PRIVATEclearpageifneeded
   }
   \abntex@ifnotempty{\imprimirepigrafe}{
      \PRIVATEbookmarkthis{\epigraphname}
      \vspace*{\fill}%
      \hspace*{.45\textwidth}
      \begin{minipage}{.5\textwidth}
         \imprimirepigrafe
      \end{minipage}
      \PRIVATEclearpageifneeded
   }
}
\makeatother

\usepackage[num]{abntex2cite} % Citações padrão ABNT, em ordem alfabética.
\citebrackets[]          % Citações com coxetes.

% ---
% Impressão da Capa
\renewcommand{\imprimircapa}{%
   \begin{capa}%
      \center
      \MakeUppercase{\imprimirinstituicao}

      \vspace{2em}
      \imprimirautor

      \vfill
      \MakeUppercase{\textbf{\imprimirtitulo}}
      \vfill

      \imprimirlocal

      \imprimirdata

      \vspace*{1cm}
   \end{capa}
}
% ---


% ---
% Folha de rosto
%   usar \imprimirfolhaderosto* caso deseje imprimir algo no verso da
%   página no caso de estar no modo twoside. Util para imprimir a Ficha
%   Bibliografica. Porem, se estiver no modo oneside, a versao sem estrela
%   é identica.

% ---
% Conteudo padrao da Folha de Rosto
\makeatletter
\renewcommand{\folhaderostocontent}{
   \begin{center}

      \MakeUppercase{\imprimirautor}

      \vfill
      \textbf{\imprimirtitulo}
      \vfill

      \abntex@ifnotempty{\imprimirpreambulo}{
      \hspace{.45\textwidth}
      \begin{minipage}{.5\textwidth}
         \imprimirpreambulo

         \vspace{2em}
         \imprimirorientadorRotulo~\imprimirorientador\par
         \abntex@ifnotempty{\imprimircoorientador}{
         \imprimircoorientadorRotulo~\imprimircoorientador
         }
      \end{minipage}
      }

      \vfill

      \imprimirlocal

      \imprimirdata
      \vspace*{1cm}

   \end{center}
}
\makeatother

%  Encoding: UTF-8
%
% %%%%%%%%%%%%%%%%%%%%%%%%%%%%%%%%%%%%%%%%%%%%%%%%%%%%%%%%%%%%%%%%%%%%%%%%%%%% %
%  Copyright (C) 2023-2025 I. F. F. dos SANTOS
%
%  Permission is hereby granted, free of charge, to any person obtaining a copy 
%  of this software and associated documentation files (the “Software”), to 
%  deal in the Software without restriction, including without limitation the 
%  rights to use, copy, modify, merge, publish, distribute, sublicense, and/or
%  sell copies of the Software, and to permit persons to whom the Software is 
%  furnished to do so, subject to the following conditions:
%
%  The above copyright notice and this permission notice shall be included in 
%  all copies or substantial portions of the Software.
%
%  THE SOFTWARE IS PROVIDED “AS IS”, WITHOUT WARRANTY OF ANY KIND, EXPRESS OR 
%  IMPLIED, INCLUDING BUT NOT LIMITED TO THE WARRANTIES OF MERCHANTABILITY, 
%  FITNESS FOR A PARTICULAR PURPOSE AND NONINFRINGEMENT. IN NO EVENT SHALL THE 
%  AUTHORS OR COPYRIGHT HOLDERS BE LIABLE FOR ANY CLAIM, DAMAGES OR OTHER 
%  LIABILITY, WHETHER IN AN ACTION OF CONTRACT, TORT OR OTHERWISE, ARISING 
%  FROM, OUT OF OR IN CONNECTION WITH THE SOFTWARE OR THE USE OR OTHER DEALINGS 
%  IN THE SOFTWARE.
% %%%%%%%%%%%%%%%%%%%%%%%%%%%%%%%%%%%%%%%%%%%%%%%%%%%%%%%%%%%%%%%%%%%%%%%%%%%% %

% Usar font 'TeX Gyre Termes' como fonte principal
\usepackage{tgtermes}

% Usar 'DejaVu Sans Mono' para \ttfamily
% https://tug.org/FontCatalogue/typewriterfonts.html
\usepackage[scaled=0.8]{DejaVuSansMono}

\usepackage[T1]{fontenc}        % Compilar correctamente com pdflatex
\usepackage[utf8]{inputenc}     % Codificacao do documento (conversão automática dos acentos).
\usepackage{ucs}                % Complemento do anterior.

% Fontes para símbolos diversos
\usepackage{
   amssymb,
   mathtools,
   pifont,
   fontawesome5,
}

% Fontes para matemática
% https://www.ctan.org/pkg/mathalpha
\usepackage{mathalpha}

\usepackage{newunicodechar}

% EOF

%  Encoding: UTF-8
%
% %%%%%%%%%%%%%%%%%%%%%%%%%%%%%%%%%%%%%%%%%%%%%%%%%%%%%%%%%%%%%%%%%%%%%%%%%%%% %
%  Copyright (C) 2023 I. F. F. dos SANTOS
%
%  Permission is hereby granted, free of charge, to any person obtaining a copy 
%  of this software and associated documentation files (the “Software”), to 
%  deal in the Software without restriction, including without limitation the 
%  rights to use, copy, modify, merge, publish, distribute, sublicense, and/or
%  sell copies of the Software, and to permit persons to whom the Software is 
%  furnished to do so, subject to the following conditions:
%
%  The above copyright notice and this permission notice shall be included in 
%  all copies or substantial portions of the Software.
%
%  THE SOFTWARE IS PROVIDED “AS IS”, WITHOUT WARRANTY OF ANY KIND, EXPRESS OR 
%  IMPLIED, INCLUDING BUT NOT LIMITED TO THE WARRANTIES OF MERCHANTABILITY, 
%  FITNESS FOR A PARTICULAR PURPOSE AND NONINFRINGEMENT. IN NO EVENT SHALL THE 
%  AUTHORS OR COPYRIGHT HOLDERS BE LIABLE FOR ANY CLAIM, DAMAGES OR OTHER 
%  LIABILITY, WHETHER IN AN ACTION OF CONTRACT, TORT OR OTHERWISE, ARISING 
%  FROM, OUT OF OR IN CONNECTION WITH THE SOFTWARE OR THE USE OR OTHER DEALINGS 
%  IN THE SOFTWARE.
% %%%%%%%%%%%%%%%%%%%%%%%%%%%%%%%%%%%%%%%%%%%%%%%%%%%%%%%%%%%%%%%%%%%%%%%%%%%% %

\PassOptionsToPackage{
   english, french, spanish,   % Idioma para hifenização.
   main=portuguese             % Idioma do documento
}{babel}
\PassOptionsToClass{
   12pt,
   a4paper,
   sumario = tradicional
}{memoir}



\usepackage{indentfirst}        % Indenta o primeiro parágrafo
\usepackage{microtype}          % para melhorias de justificação.
\usepackage[htt]{hyphenat}      % Permitir hifenização usando \texttt
\usepackage[                    % Margens conforme padrão ABNT
   top = 3cm,
   left = 3cm,
   bottom = 2cm,
   right = 2cm
]{geometry}

\usepackage{fancyhdr}

\lhead[\thepage]{\thechapter}
\chead[]{}
\rhead[\thesection]{\thepage}

\lfoot[]{}
\cfoot[]{}
\rfoot[]{}

\renewcommand{\headrulewidth}{1.5pt}
\renewcommand{\footrulewidth}{0pt}


% 1ex - É a largura da letra x (minúscula)
% 1em - É a largura da letra M (Maiúscula)
\setlength{\parindent}{3ex}         % Tamanho do parágrafo
\setlength{\parskip}{0em}           % Espaço entre parágrafos

\makeatletter
\expandafter\def\expandafter\normalsize\expandafter{%
    \normalsize
    \setlength{\abovedisplayskip}{0em}
    \setlength{\belowdisplayskip}{0em}
    \setlength{\abovedisplayshortskip}{0em}
    \setlength{\belowdisplayshortskip}{0em}
}
\makeatother

% EOF

\RequirePackage{listings}

\RequirePackage{tikz, tikz-cd}      % Insercao de imagens configuraveis.

\usetikzlibrary{shapes.geometric, arrows}
\tikzstyle{startstop} = [rectangle, rounded corners, minimum width=3cm, minimum height=1cm,text centered, draw=black]
\tikzstyle{io} = [trapezium, trapezium left angle=70, trapezium right angle=110, minimum width=3cm, minimum height=1cm, text centered, draw=black]
\tikzstyle{process} = [rectangle, minimum width=3cm, minimum height=1cm, text centered, draw=black]
\tikzstyle{decision} = [diamond, minimum width=3cm, minimum height=1cm, text centered, draw=black]
\tikzstyle{conector} = [circle, minimum width=0.5cm, minimum height=0.5cm, text centered, draw=black]
\tikzstyle{arrow} = [thick,->,>=stealth]
\tikzstyle{line} = [draw, -latex']

% see https://latexcolor.com/
\definecolor{codegreen}{rgb}{0,0.6,0}
\definecolor{codegray}{rgb}{0.5,0.5,0.5}
\definecolor{amaranth}{rgb}{0.9, 0.17, 0.31}
\definecolor{theBlack}{gray}{0.0}
\definecolor{theWhite}{gray}{0.9}

\RequirePackage{caption}
\DeclareCaptionFont{white}{\color{white}}
\DeclareCaptionFormat{listing}{\hspace*{-0.4pt}\colorbox{gray}{\parbox{\textwidth}{#1#2#3}}}
\captionsetup[lstlisting]{format=listing,labelfont=white,textfont=white}

\renewcommand{\lstlistingname}{Código}

\lstset{
   backgroundcolor=\color{gray!5!white},   % choose the background color; you must add \usepackage{color} or \usepackage{xcolor}; should come as last argument
   basicstyle=\mdseries\ttfamily\footnotesize\color{black}, % the size of the fonts that are used for the code
   breakatwhitespace=false,         % sets if automatic breaks should only happen at whitespace
   breaklines=true,                 % sets automatic line breaking
   captionpos=t,                    % sets the caption-position to bottom
   columns=fixed,                   % Using fixed column width (for e.g. nice alignment)
   escapechar=µ,
   numbers=none,                    % Not use numbers
   keepspaces=true,                 % keeps spaces in text, useful for keeping indentation of code (possibly needs columns=flexible)
   showspaces=false,                % show spaces everywhere adding particular underscores; it overrides 'showstringspaces'
   showstringspaces=false,          % underline spaces within strings only
   showtabs=false,                  % show tabs within strings adding particular underscores
   tabsize=3, 	                     % sets default tabsize to 3 spaces
   %frame=single,	                  % adds a frame around the code
   %rulecolor=\color{black},         % if not set, the frame-color may be changed on line-breaks within not-black text (e.g. comments (green here))
}

\lstdefinestyle{c}{
   language=C, % the language of the code (can be overrided per snippet)
   commentstyle=\color{codegray}, % comment style
   keywordstyle={\color{codegreen}\bfseries},
   stringstyle=\color{amaranth} % string literal style
}

\lstdefinestyle{f90}{
   language=Fortran, % the language of the code (can be overrided per snippet)
   commentstyle=\color{codegray}, % comment style
   keywordstyle={\color{codegreen}\bfseries},
   stringstyle=\color{amaranth} % string literal style
}

\lstdefinestyle{java}{
   language=Java, % the language of the code (can be overrided per snippet)
   commentstyle=\color{codegray}, % comment style
   keywordstyle={\color{codegreen}\bfseries},
   stringstyle=\color{amaranth} % string literal style
}

\lstdefinestyle{py}{
   language=Python, % the language of the code (can be overrided per snippet)
   commentstyle=\color{codegray}, % comment style
   keywordstyle={\color{codegreen}\bfseries},
   stringstyle=\color{amaranth} % string literal style
}

\lstdefinestyle{gnuplot}{
   language=Gnuplot, % the language of the code (can be overrided per snippet)
   commentstyle=\color{codegray}, % comment style
   keywordstyle={\color{codegreen}\bfseries},
   stringstyle=\color{amaranth} % string literal style
}

\lstdefinestyle{bash}{
   language=bash, % the language of the code (can be overrided per snippet)
   commentstyle=\color{codegray}, % comment style
   keywordstyle={\color{codegreen}\bfseries},
   stringstyle=\color{amaranth}, % string literal style
   morekeywords={apt, pacman, yum, zypper, dnf, dns, mkdir, cp, configure, make, tar},
}

\lstdefinestyle{pseudo}{
   commentstyle=\color{codegray}, % comment style
   keywordstyle={\color{codegreen}\bfseries},
   stringstyle=\color{amaranth}, % string literal style
   morekeywords={inicio, fim, para, ate, subrotinas, se, senao, entao, funcao, inteiro, enquanto, gnuplot, logico, verdadeiro, subrotina, variavel, inicialize},
   morestring=[b]",
   morecomment={[l]//},
   inputencoding=utf8,
   extendedchars=true,
   literate=%
      {á}{{\'a}}1 {é}{{\'e}}1 {í}{{\'i}}1 {ó}{{\'o}}1 {ú}{{\'u}}1
      {Á}{{\'A}}1 {É}{{\'E}}1 {Í}{{\'I}}1 {Ó}{{\'O}}1 {Ú}{{\'U}}1
      {à}{{\`a}}1 {è}{{\`e}}1 {ì}{{\`i}}1 {ò}{{\`o}}1 {ù}{{\`u}}1
      {À}{{\`A}}1 {È}{{\'E}}1 {Ì}{{\`I}}1 {Ò}{{\`O}}1 {Ù}{{\`U}}1
      {ä}{{\"a}}1 {ë}{{\"e}}1 {ï}{{\"i}}1 {ö}{{\"o}}1 {ü}{{\"u}}1
      {Ä}{{\"A}}1 {Ë}{{\"E}}1 {Ï}{{\"I}}1 {Ö}{{\"O}}1 {Ü}{{\"U}}1
      {â}{{\^a}}1 {ê}{{\^e}}1 {î}{{\^i}}1 {ô}{{\^o}}1 {û}{{\^u}}1
      {Â}{{\^A}}1 {Ê}{{\^E}}1 {Î}{{\^I}}1 {Ô}{{\^O}}1 {Û}{{\^U}}1
      {ã}{{\~a}}1 {ẽ}{{\~e}}1 {ĩ}{{\~i}}1 {õ}{{\~o}}1 {ũ}{{\~u}}1
      {Ã}{{\~A}}1 {Ẽ}{{\~E}}1 {Ĩ}{{\~I}}1 {Õ}{{\~O}}1 {Ũ}{{\~U}}1
      {œ}{{\oe}}1 {Œ}{{\OE}}1 {æ}{{\ae}}1 {Æ}{{\AE}}1 {ß}{{\ss}}1
      {ű}{{\H{u}}}1 {Ű}{{\H{U}}}1 {ő}{{\H{o}}}1 {Ő}{{\H{O}}}1
      {ç}{{\c c}}1 {Ç}{{\c C}}1 {ø}{{\o}}1 {å}{{\r a}}1 {Å}{{\r A}}1
      {€}{{\euro}}1 {£}{{\pounds}}1 {«}{{\guillemotleft}}1
      {»}{{\guillemotright}}1 {ñ}{{\~n}}1 {Ñ}{{\~N}}1 {¿}{{?`}}1 {¡}{{!`}}1
      {←}{{$\leftarrow$}}1
      {!=}{{$\neq$}}1
}

\makeatletter
% \expandafter for the case that the filename is given in a command
\newcommand{\replunderscores}[1]{\expandafter\@repl@underscores#1_\relax}

\def\@repl@underscores#1_#2\relax{%
    \ifx \relax #2\relax
        % #2 is empty => finish
        #1%
    \else
        % #2 is not empty => underscore was contained, needs to be replaced
        #1%
        \textunderscore
        % continue replacing
        % #2 ends with an extra underscore so I don't need to add another one
        \@repl@underscores#2\relax
    \fi
}
\makeatother

\RequirePackage[most]{tcolorbox}
\RequirePackage{fontawesome5}       % Permite o uso da fonte Font Awesome

\newcommand{\inserircodigo}[2]{
   \begin{tcolorbox}[
      colback=gray!5!white,
      colframe=gray!75!black,
      title=\ttfamily\replunderscores{#2}
   ]
      \lstinputlisting[style = #1]{./SourceCodes/#2}
   \end{tcolorbox}
}


\usepackage{hyperref, bookmark}         % Insercao de hiperreferencias.
\usepackage{xcolor, color}              % hiperreferencias coloridas
\usepackage[hyphenbreaks]{breakurl}     % Quebra linha nno comando \url
\usepackage{url}                        % Bibliografia padrão ABNT.

% Por praticidade, em textos não academicos, prefiro usar como segue
%\RequirePackage[num, overcite]{abntex2cite} % Bibliografia padrão ABNT.
%\citebrackets[]                             % Citações com colchetes.

\hypersetup{
   pdfcreator={LaTeX with abnTeX2},
   bookmarksdepth=4,
   pagebackref=true,
   breaklinks=true,
   % Caso o documento seja impresso, marque a opção abaixo como false,
   % caso contrário, recomendo que matenha como true.
   colorlinks=true,             % false: boxed links; true: colored links
   linkcolor=blue(ryb),         % cor das ligações internas
   citecolor=darkmidnightblue,  % cor das chamadas bibliográficas
   urlcolor=teal,               % cor dos URL
   linkbordercolor=0.01 .28 1,  % cor das ligações internas
   citebordercolor=0 .2 .4,     % cor das chamadas bibliográficas
   urlbordercolor=0 .5 .5,      % cor dos URL
}
%  Configuração do PDF
\AtBeginDocument{
   \hypersetup{
      pdftitle={\imprimirtitulo}, 
      pdfauthor={\imprimirautor},
   }
}

%  Encoding: UTF-8
%
% %%%%%%%%%%%%%%%%%%%%%%%%%%%%%%%%%%%%%%%%%%%%%%%%%%%%%%%%%%%%%%%%%%%%%%%%%%%% %
%  Copyright (C) 2023 I. F. F. dos SANTOS
%
%  Permission is hereby granted, free of charge, to any person obtaining a copy 
%  of this software and associated documentation files (the “Software”), to 
%  deal in the Software without restriction, including without limitation the 
%  rights to use, copy, modify, merge, publish, distribute, sublicense, and/or
%  sell copies of the Software, and to permit persons to whom the Software is 
%  furnished to do so, subject to the following conditions:
%
%  The above copyright notice and this permission notice shall be included in 
%  all copies or substantial portions of the Software.
%
%  THE SOFTWARE IS PROVIDED “AS IS”, WITHOUT WARRANTY OF ANY KIND, EXPRESS OR 
%  IMPLIED, INCLUDING BUT NOT LIMITED TO THE WARRANTIES OF MERCHANTABILITY, 
%  FITNESS FOR A PARTICULAR PURPOSE AND NONINFRINGEMENT. IN NO EVENT SHALL THE 
%  AUTHORS OR COPYRIGHT HOLDERS BE LIABLE FOR ANY CLAIM, DAMAGES OR OTHER 
%  LIABILITY, WHETHER IN AN ACTION OF CONTRACT, TORT OR OTHERWISE, ARISING 
%  FROM, OUT OF OR IN CONNECTION WITH THE SOFTWARE OR THE USE OR OTHER DEALINGS 
%  IN THE SOFTWARE.
% %%%%%%%%%%%%%%%%%%%%%%%%%%%%%%%%%%%%%%%%%%%%%%%%%%%%%%%%%%%%%%%%%%%%%%%%%%%% %

\usepackage{lipsum}    %
\usepackage{lettrine}  % Letra capitular
\usepackage{enumitem}  %
\setlist[enumerate]{nosep, topsep=-\parskip}
\setlist[itemize]{nosep, topsep=-\parskip}

\usepackage{graphicx}                  % Inclusão de figuras.
\usepackage{float}                     % Para posicionar figuras corretamente.
\usepackage{subcaption}                % Para legendas nas subfiguras.
\usepackage[cleanup]{gnuplottex}       % Para uso do GNUPlotTex.
\usepackage[tickmarkheight=5pt]{todonotes}                 % Para TODO

% EOF



% --------------------------------------
% Informações de dados para CAPA e FOLHA DE ROSTO
% --------------------------------------
% Nesta parte ainda não são permitidos acentos, segue um dicionário
% cedilha ç = \c{c}, Ç = \c{C}
% agudo á = \'{a}, Á = \'{A}
% agudo é = \'{e}, É = \'{E}
% agudo í = \'{i}, Í = \'{I}
% agudo ó = \'{o}, Ó = \'{O}
% agudo ú = \'{u}, Ú = \'{U}
\titulo{%
   Tuim Environments \\ \small{Version 0}%
}
\autor{%
   I. F. F. dos{ }SANTOS\thanks{\href{mailto:ismaellxd@gmail.com}{ismaellxd@gmail.com}}%
}
\local{Maceió}

% informações do PDF
\hypersetup{
   english,
   pdftitle={\imprimirtitulo}, 
   pdfauthor={\imprimirautor},
   pdfcreator={LaTeX with abnTeX2},
   %pagebackref=true,
   bookmarksdepth=4,
   colorlinks=true,            % false: boxed links; true: colored links
   linkcolor=blue,             % color of internal links
   citecolor=blue,             % color of links to bibliography
   filecolor=mageta,          % color of file links
   urlcolor=blue
}
\makeindex

\begin{document}
\selectlanguage{english}
% ------------------------------------------------------------------------------
% ELEMENTOS PRÉ-TEXTUAIS
% ------------------------------------------------------------------------------
\maketitle

\begin{resumoumacoluna}
   This documents describes the Application Binary Interface (ABI)
   for Tuim Environments\cite{tuim},
   the Command Line Interface (CLI)
   to build and execute software in those invironments
   and the Application Programing Interface (API)
   to develop software for for those environment.

   This document is in active development.
   Its CLI and ABI might change any time without any notice.
\end{resumoumacoluna}

%\begin{resumoumacoluna}
%   This document describes command line tools and standard libraries
%   to build application for Tuim\cite{tuim} environments.
%   The interface is designed to be cross-platform and
%   to allow cross-compilation and interoperability betewen distinct
%   languages.
%   Generated code is designed to mach the System V ABI\cite{systemv},
%   to generate position independent code and to allow binary portability,
%   even betewen distincts operational systems.

%   This document is in active development.
%   Its CLI and ABI might change any time without any notice.
%\end{resumoumacoluna}

\newpage
\tableofcontents
\newpage
% ------------------------------------------------------------------------------
% ELEMENTOS TEXTUAIS
% ------------------------------------------------------------------------------
\textual

% -------------------------------------
% SECTION
% -------------------------------------
\section{Introduction}

Tuim is designed to provide a way to execute the same binary in distincts systems
without the need of recompilation or virtual machines.
A Tuim Environment consists of a user interface suitable to start the execution
of softwares conforming to specifications in this document
and the system in here the execution takes place.

% -----------------
% SUBSECTION
% -----------------
\subsection{Application Binary Interface}

Tuim is designed to execute the
Executable and Linkable Format (ELF),
from System V\cite{systemv} ABI.
The \texttt{EI\_OSABI} field in ELF identification
must holds \texttt{ELFOSABI\_NONE},
ensuring that no ELF extension are used excepts those defined in
processor supplement.

Both executable and shared objects must be build
with position independent code in order to allow Tuim
to load correctly the code.
Tuim also preserves in the memory the
difference between virtual addresses in the file.

By convenction executable files uses the \texttt{.elf} extension,
while \texttt{.so} is for shared objects and
\texttt{.o} for relocatable files.

% -----------------
% SUBSECTION
% -----------------
\subsection{Tuim from command-line interface}

For command-line environments Tuim can be used to load and execute
a ELF via:

\begin{lstlisting}[style=bash]
tuim program.elf [arg ...]
\end{lstlisting}

Then the \texttt{program.elf} executable file is loaded,
if the file request a library then the unique default directory
where the loader find for it is
the subdirectory \texttt{lib} at the directory from the
environment variable \texttt{TUIM\_HOME}.
If all dependencies are present and symbols are correctly
relocated then execution takes place.

The executable file's entry point is called like if it has
the following signature in a C program.

\begin{lstlisting}[style=c]
int main(int argc, char **argv);
\end{lstlisting}

% -----------------
% SUBSECTION
% -----------------
\subsection{Tuim from graphical user interface}

\textbf{TODO}

\newpage
% -------------------------------------
% SECTION
% -------------------------------------
\section{Runtime libraries}

A program may request libraries as shared objects,
if the shared object name has no slash then the loader find for it
first on directories specified by the environment variable
\texttt{LD\_LIBRARY\_PATH} and then in \texttt{<TUIM\_HOME>/lib}.
The program may need the environment variable \texttt{LD\_LIBRARY\_PATH}
set in order to allow the loader dinamically link with custom shared objects.
The directory \texttt{<TUIM\_HOME>/lib} is reserved for Tuim
and all shared object names are reserved for Tuim,
this means that if a program request any shared object for purposes
others that defined in this document than the environment variable
\texttt{LD\_LIBRARY\_PATH} need to be used.

% -----------------
% SUBSECTION
% -----------------
\subsection{Compiler runtime library}

The \texttt{libcrt.so} shared object
expose symbols from the Compiler Runtime Library,
usually helper functions that the compiler need.
Symbols exposed by this shared object
begins with two underscores or an underscore followed by a capital letter.

Some processor suplements provide a specification for the
Compiler Runtime Library, in that case this is the library in the shared
object \texttt{libcrt.so}.
If the processor suplement do not provide such specification than
is a undefined bahavior the explicit use of
any symbol exposed by \texttt{libcrt.so} in source code,
these symbols are reserved for the use of the compiler.

TODO: Alternatively, I want to provide a CLang compatible ABI for processors
without such specification.

% -----------------
% SUBSECTION
% -----------------
\subsection{The C standard library}

The \texttt{libc.so} shared object
expose symbols from the C Standard Library, the latest version\cite{ISO9899}.
Symbols exposed by this shared object are symbols reserved by the C standard.

\newpage
% -------------------------------------
% SECTION
% -------------------------------------
\section{Tuim Developer Kit}

The Tuim Developer Kit is a set of tools designed to write program for Tuim
and to allow language interoperability.

% -----------------
% SUBSECTION
% -----------------
\subsection{Linker}

\noindent NAME

\begin{lstlisting}[style=bash]
ld - The Tuim linker
\end{lstlisting}

\noindent SYNOPSIS

\begin{lstlisting}[style=bash]
ld [OPTIONS] objfile ...
\end{lstlisting}

\noindent DESCRIPTION

The linker is a sofware to link relocatable objects,
it outputs executables or a shared objects.

\noindent OPTIONS

\begin{description}[style=multiline,leftmargin=5cm]
   \item[\texttt{{-}{-}version}]
   Show the TDK versions, the default target and exit.
   \item[\texttt{{-}{-}shared}]
   Enable generation of shared object,
   if not present defaults to generation of executable.
   \item[\texttt{-e <symbol>}]
   Define the \texttt{<symbol>} as the entry point
   (suitable only for executables),
   if not present defaults to \texttt{main}.
   \item[\texttt{-o <name>}]
   Define the output file's name,
   if not present defaults to the same input name but with
   extension changed to \texttt{elf} (for executables)
   or \texttt{so} (for shared objects).
   \item[\texttt{-l<library>}]
   Add a \texttt{DT\_NEEDED} entry on the dynamic table for the
   \texttt{lib<library>.so} dependency.
\end{description}

% -----------------
% SUBSECTION
% -----------------
\subsection{Assembler}

\noindent NAME

\begin{lstlisting}[style=bash]
as - The Tuim assembler
\end{lstlisting}

\noindent SYNOPSIS

\begin{lstlisting}[style=bash]
as [OPTIONS] asmfile ...
\end{lstlisting}

\noindent DESCRIPTION

The assembler is a software to assemble human readable machine code,
it outputs relocatable files.

\noindent OPTIONS

\begin{description}[style=multiline,leftmargin=5cm]
   \item[\texttt{{-}{-}version}]
   Show the TDK version and the default target and exit.
   \item[\texttt{-o <name>}]
   Define the output file's name,
   if not present defaults to the same input name but with
   extension changed to \texttt{o}.
   \item[\texttt{{-}{-}target <target>}]
   Specify the ISA.
\end{description}

% -----------------
% SUBSECTION
% -----------------
\subsection{C Compiler}

\noindent NAME

\begin{lstlisting}[style=bash]
cc - The Tuim C compiler
\end{lstlisting}

\noindent SYNOPSIS

\begin{lstlisting}[style=bash]
cc [OPTIONS] srcfile ...
\end{lstlisting}

\noindent DESCRIPTION

The C compiler is a software to compile source code writter in the
C Programing Language,
it outputs relocatable files.

\noindent OPTIONS

\begin{description}[style=multiline,leftmargin=5cm]
   \item[\texttt{{-}{-}version}]
   Show the TDK version and the default target and exit.
   \item[\texttt{-o <name>}]
   Define the output file's name,
   if not present defaults to the same input name but with
   extension changed to \texttt{o}(for relocatable objects).
   \item[\texttt{-I <path>}]
   Define the \texttt{<path>} for find header files.
   \item[\texttt{{-}{-}target <target>}]
   Define the target processor.
\end{description}

% -----------------
% SUBSECTION
% -----------------
\subsection{Build system}

\noindent NAME

\begin{lstlisting}[style=bash]
mk - The Tuim utility to build projects
\end{lstlisting}

\noindent SYNOPSIS

\begin{lstlisting}[style=bash]
mk [OPTIONS] yamlfile ...
\end{lstlisting}

\noindent DESCRIPTION

The build system is a software to automate some processes when building
executables and shared objects from multiple source files.

\noindent OPTIONS

\begin{description}[style=multiline,leftmargin=5cm]
   \item[\texttt{{-}{-}version}]
   Show the TDK version and the default target and exit.
   \item[\texttt{{-}{-}target <target>}]
   Define the target processor.
\end{description}

% -----------------
% SUBSECTION
% -----------------
\subsection{Configuration file}

Configuration files are used to describe how source code are assembled,
compiled and linked to generate a executable or shared object.
The configuration file entries are:

\begin{description}[style=multiline,leftmargin=5cm]
   \item[\texttt{name}]
   The name of the output.
   Output name is exact the value in this field for executables,
   but for shared objects the output name is \texttt{lib<value>.so}.
   \item[\texttt{object\_type}]
   The type of object file to generate.
   Allowed values: \texttt{executable}, \texttt{shared}.
   \item[\texttt{dependencies}]
   List of dependency libraries for the dynamic table.
   \item[\texttt{source}]
   List of sources (see bellow).
\end{description}

Source code files are describeded by the list \texttt{source}:

\begin{description}[style=multiline,leftmargin=5cm]
   \item[\texttt{language}]
   The programing language the source entry.
   If not present then the source code are assembled.
   Allowed value: \texttt{C}.
   \item[\texttt{target}]
   If present then define the target that the source entry is for,
   if target command line option do not mach this value than
   the entry is ignored.
   For supported values see the appendex \ref{apendice:target}.
   \item[\texttt{files}]
   A list of source code files to be compiled/assembled,
   file names are relative to the configuration file location.
\end{description}


% ------------------------------------------------------------------------------
% ELEMENTOS PÓS-TEXTUAIS
% ------------------------------------------------------------------------------
   \postextual
   % ---------------------------------------------------------------------------
   % Referências bibliográficas
   % ---------------------------------------------------------------------------
   \bibliography{references}
   \apendices
   \renewcommand{\thesection}{\Alph{section}}

\section{Targets}
\label{apendice:target}

Allowed targets are
\begin{itemize}
   \item \texttt{x86}
   \item \texttt{x86\_64}
   \item \texttt{arm}
   \item \texttt{aarch64}
   \item \texttt{riscv32}
   \item \texttt{riscv64}
\end{itemize}

% -----------------
% SUBSECTION
% -----------------
\subsection{Example}

Consider the following project.

\begin{lstlisting}
project
`-- src
    |-- boo.asm
    |-- foo.c
    `-- project.yaml
\end{lstlisting}

Where the contents of \texttt{project.yaml} is

\begin{lstlisting}[style=yaml]
name: foo
type: shared
dependencies:
  - libc.so
  - libcrt.so
source:
  - language: C
    files:
      - foo.c
  - target: arm
    files:
      - boo.asm
\end{lstlisting}

\noindent
Therefore the command to build it is

\begin{lstlisting}[style=bash]
mk src/project.yaml
\end{lstlisting}

\noindent
And the result is like if executing the commands

\begin{lstlisting}[style=bash]
# On ARM machine
cc -o tmp/foo.o src/foo.c
as -o tmp/boo.o src/boo.asm
ld -o lib/foo.so -lc -lcrt tmp/foo.o tmp/boo.o
# On non ARM machine
cc -o tmp/foo.o src/foo.c
ld -o lib/foo.so -lc -lcrt tmp/foo.o
\end{lstlisting}

\end{document}
