%   Copyright (c) 2024-2025 I.F.F. dos SANTOS <ismaellxd@gmail.com>

%   Permission is hereby granted, free of charge, to any person obtaining a copy 
%   of this software and associated documentation files (the “Software”), to 
%   deal in the Software without restriction, including without limitation the 
%   rights to use, copy, modify, merge, publish, distribute, sublicense, and/or 
%   sell copies of the Software, and to permit persons to whom the Software is 
%   furnished to do so, subject to the following conditions:

%   The above copyright notice and this permission notice shall be included in 
%   all copies or substantial portions of the Software.

%   THE SOFTWARE IS PROVIDED “AS IS”, WITHOUT WARRANTY OF ANY KIND, EXPRESS OR 
%   IMPLIED, INCLUDING BUT NOT LIMITED TO THE WARRANTIES OF MERCHANTABILITY, 
%   FITNESS FOR A PARTICULAR PURPOSE AND NONINFRINGEMENT. IN NO EVENT SHALL THE 
%   AUTHORS OR COPYRIGHT HOLDERS BE LIABLE FOR ANY CLAIM, DAMAGES OR OTHER 
%   LIABILITY, WHETHER IN AN ACTION OF CONTRACT, TORT OR OTHERWISE, ARISING 
%   FROM, OUT OF OR IN CONNECTION WITH THE SOFTWARE OR THE USE OR OTHER DEALINGS 
%   IN THE SOFTWARE.

\documentclass[
   % -- opções da classe memoir --
   12pt,                         % tamanho da fonte.
   openright,                    % capítulos começam em pág ímpar (insere página vazia caso preciso).
   twoside,                      % para impressão em recto. Oposto a twoside.
   a4paper,                      % tamanho do papel. 
   sumario = tradicional,        % Para indentar os sumários
   % -- opções do pacote babel --
   english,                      % idioma adicional para hifenização.
   xcolor=table                  % Tabelas coloridadas.
]{abntex2}
% --------------------------------------
% PACOTES
% --------------------------------------
%  Encoding: UTF-8
%
% %%%%%%%%%%%%%%%%%%%%%%%%%%%%%%%%%%%%%%%%%%%%%%%%%%%%%%%%%%%%%%%%%%%%%%%%%%%% %
%  Copyright (C) 2023 I. F. F. dos SANTOS
%
%  Permission is hereby granted, free of charge, to any person obtaining a copy 
%  of this software and associated documentation files (the “Software”), to 
%  deal in the Software without restriction, including without limitation the 
%  rights to use, copy, modify, merge, publish, distribute, sublicense, and/or
%  sell copies of the Software, and to permit persons to whom the Software is 
%  furnished to do so, subject to the following conditions:
%
%  The above copyright notice and this permission notice shall be included in 
%  all copies or substantial portions of the Software.
%
%  THE SOFTWARE IS PROVIDED “AS IS”, WITHOUT WARRANTY OF ANY KIND, EXPRESS OR 
%  IMPLIED, INCLUDING BUT NOT LIMITED TO THE WARRANTIES OF MERCHANTABILITY, 
%  FITNESS FOR A PARTICULAR PURPOSE AND NONINFRINGEMENT. IN NO EVENT SHALL THE 
%  AUTHORS OR COPYRIGHT HOLDERS BE LIABLE FOR ANY CLAIM, DAMAGES OR OTHER 
%  LIABILITY, WHETHER IN AN ACTION OF CONTRACT, TORT OR OTHERWISE, ARISING 
%  FROM, OUT OF OR IN CONNECTION WITH THE SOFTWARE OR THE USE OR OTHER DEALINGS 
%  IN THE SOFTWARE.
% %%%%%%%%%%%%%%%%%%%%%%%%%%%%%%%%%%%%%%%%%%%%%%%%%%%%%%%%%%%%%%%%%%%%%%%%%%%% %

\renewcommand{\ABNTEXchapterfont}{\bfseries\rmfamily}
\renewcommand{\ABNTEXsectionfont}{\bfseries\rmfamily}
\renewcommand{\ABNTEXsubsectionfont}{\bfseries\rmfamily}
\renewcommand{\ABNTEXsubsubsectionfont}{\bfseries\rmfamily}

\renewcommand{\ABNTEXchapterfontsize}{\huge}
\renewcommand{\ABNTEXsectionfontsize}{\LARGE}
\renewcommand{\ABNTEXsubsectionfontsize}{\large}
\renewcommand{\ABNTEXsubsubsectionfontsize}{\normalsize}

\tituloestrangeiro{\vspace{-2em}}
\setlength{\ABNTEXsignwidth}{12cm}

\renewenvironment{dedicatoria}{}{}
\renewenvironment{agradecimentos}{}{}
\renewenvironment{epigrafe}{}{}

\providecommand{\imprimirdedicatoria}{}
\providecommand{\imprimiragradecimentos}{}
\providecommand{\imprimirepigrafe}{}

\renewcommand{\imprimirdata}{\today}

\renewcommand{\dedicatoria}[1]{\renewcommand{\imprimirdedicatoria}{#1}}
\renewcommand{\agradecimentos}[1]{\renewcommand{\imprimiragradecimentos}{#1}}
\renewcommand{\epigrafe}[1]{\renewcommand{\imprimirepigrafe}{#1}}


\makeatletter
\providecommand{\pretextualpreconfigurado}{
   \abntex@ifnotempty{\imprimirdedicatoria}{
      \PRIVATEbookmarkthis{\dedicatorianame}
      \vspace*{\fill}%
      \hspace*{.45\textwidth}
      \begin{minipage}{.5\textwidth}
         \imprimirdedicatoria
      \end{minipage}
      \PRIVATEclearpageifneeded
   }
   \abntex@ifnotempty{\imprimiragradecimentos}{
      \pretextualchapter{\agradecimentosname}
      \imprimiragradecimentos
      \PRIVATEclearpageifneeded
   }
   \abntex@ifnotempty{\imprimirepigrafe}{
      \PRIVATEbookmarkthis{\epigraphname}
      \vspace*{\fill}%
      \hspace*{.45\textwidth}
      \begin{minipage}{.5\textwidth}
         \imprimirepigrafe
      \end{minipage}
      \PRIVATEclearpageifneeded
   }
}
\makeatother

\usepackage[num]{abntex2cite} % Citações padrão ABNT, em ordem alfabética.
\citebrackets[]          % Citações com coxetes.

% ---
% Impressão da Capa
\renewcommand{\imprimircapa}{%
   \begin{capa}%
      \center
      \MakeUppercase{\imprimirinstituicao}

      \vspace{2em}
      \imprimirautor

      \vfill
      \MakeUppercase{\textbf{\imprimirtitulo}}
      \vfill

      \imprimirlocal

      \imprimirdata

      \vspace*{1cm}
   \end{capa}
}
% ---


% ---
% Folha de rosto
%   usar \imprimirfolhaderosto* caso deseje imprimir algo no verso da
%   página no caso de estar no modo twoside. Util para imprimir a Ficha
%   Bibliografica. Porem, se estiver no modo oneside, a versao sem estrela
%   é identica.

% ---
% Conteudo padrao da Folha de Rosto
\makeatletter
\renewcommand{\folhaderostocontent}{
   \begin{center}

      \MakeUppercase{\imprimirautor}

      \vfill
      \textbf{\imprimirtitulo}
      \vfill

      \abntex@ifnotempty{\imprimirpreambulo}{
      \hspace{.45\textwidth}
      \begin{minipage}{.5\textwidth}
         \imprimirpreambulo

         \vspace{2em}
         \imprimirorientadorRotulo~\imprimirorientador\par
         \abntex@ifnotempty{\imprimircoorientador}{
         \imprimircoorientadorRotulo~\imprimircoorientador
         }
      \end{minipage}
      }

      \vfill

      \imprimirlocal

      \imprimirdata
      \vspace*{1cm}

   \end{center}
}
\makeatother

%  Encoding: UTF-8
%
% %%%%%%%%%%%%%%%%%%%%%%%%%%%%%%%%%%%%%%%%%%%%%%%%%%%%%%%%%%%%%%%%%%%%%%%%%%%% %
%  Copyright (C) 2023-2025 I. F. F. dos SANTOS
%
%  Permission is hereby granted, free of charge, to any person obtaining a copy 
%  of this software and associated documentation files (the “Software”), to 
%  deal in the Software without restriction, including without limitation the 
%  rights to use, copy, modify, merge, publish, distribute, sublicense, and/or
%  sell copies of the Software, and to permit persons to whom the Software is 
%  furnished to do so, subject to the following conditions:
%
%  The above copyright notice and this permission notice shall be included in 
%  all copies or substantial portions of the Software.
%
%  THE SOFTWARE IS PROVIDED “AS IS”, WITHOUT WARRANTY OF ANY KIND, EXPRESS OR 
%  IMPLIED, INCLUDING BUT NOT LIMITED TO THE WARRANTIES OF MERCHANTABILITY, 
%  FITNESS FOR A PARTICULAR PURPOSE AND NONINFRINGEMENT. IN NO EVENT SHALL THE 
%  AUTHORS OR COPYRIGHT HOLDERS BE LIABLE FOR ANY CLAIM, DAMAGES OR OTHER 
%  LIABILITY, WHETHER IN AN ACTION OF CONTRACT, TORT OR OTHERWISE, ARISING 
%  FROM, OUT OF OR IN CONNECTION WITH THE SOFTWARE OR THE USE OR OTHER DEALINGS 
%  IN THE SOFTWARE.
% %%%%%%%%%%%%%%%%%%%%%%%%%%%%%%%%%%%%%%%%%%%%%%%%%%%%%%%%%%%%%%%%%%%%%%%%%%%% %

% Usar font 'TeX Gyre Termes' como fonte principal
\usepackage{tgtermes}

% Usar 'DejaVu Sans Mono' para \ttfamily
% https://tug.org/FontCatalogue/typewriterfonts.html
\usepackage[scaled=0.8]{DejaVuSansMono}

\usepackage[T1]{fontenc}        % Compilar correctamente com pdflatex
\usepackage[utf8]{inputenc}     % Codificacao do documento (conversão automática dos acentos).
\usepackage{ucs}                % Complemento do anterior.

% Fontes para símbolos diversos
\usepackage{
   amssymb,
   mathtools,
   pifont,
   fontawesome5,
}

% Fontes para matemática
% https://www.ctan.org/pkg/mathalpha
\usepackage{mathalpha}

\usepackage{newunicodechar}

% EOF

%  Encoding: UTF-8
%
% %%%%%%%%%%%%%%%%%%%%%%%%%%%%%%%%%%%%%%%%%%%%%%%%%%%%%%%%%%%%%%%%%%%%%%%%%%%% %
%  Copyright (C) 2023 I. F. F. dos SANTOS
%
%  Permission is hereby granted, free of charge, to any person obtaining a copy 
%  of this software and associated documentation files (the “Software”), to 
%  deal in the Software without restriction, including without limitation the 
%  rights to use, copy, modify, merge, publish, distribute, sublicense, and/or
%  sell copies of the Software, and to permit persons to whom the Software is 
%  furnished to do so, subject to the following conditions:
%
%  The above copyright notice and this permission notice shall be included in 
%  all copies or substantial portions of the Software.
%
%  THE SOFTWARE IS PROVIDED “AS IS”, WITHOUT WARRANTY OF ANY KIND, EXPRESS OR 
%  IMPLIED, INCLUDING BUT NOT LIMITED TO THE WARRANTIES OF MERCHANTABILITY, 
%  FITNESS FOR A PARTICULAR PURPOSE AND NONINFRINGEMENT. IN NO EVENT SHALL THE 
%  AUTHORS OR COPYRIGHT HOLDERS BE LIABLE FOR ANY CLAIM, DAMAGES OR OTHER 
%  LIABILITY, WHETHER IN AN ACTION OF CONTRACT, TORT OR OTHERWISE, ARISING 
%  FROM, OUT OF OR IN CONNECTION WITH THE SOFTWARE OR THE USE OR OTHER DEALINGS 
%  IN THE SOFTWARE.
% %%%%%%%%%%%%%%%%%%%%%%%%%%%%%%%%%%%%%%%%%%%%%%%%%%%%%%%%%%%%%%%%%%%%%%%%%%%% %

\PassOptionsToPackage{
   english, french, spanish,   % Idioma para hifenização.
   main=portuguese             % Idioma do documento
}{babel}
\PassOptionsToClass{
   12pt,
   a4paper,
   sumario = tradicional
}{memoir}



\usepackage{indentfirst}        % Indenta o primeiro parágrafo
\usepackage{microtype}          % para melhorias de justificação.
\usepackage[htt]{hyphenat}      % Permitir hifenização usando \texttt
\usepackage[                    % Margens conforme padrão ABNT
   top = 3cm,
   left = 3cm,
   bottom = 2cm,
   right = 2cm
]{geometry}

\usepackage{fancyhdr}

\lhead[\thepage]{\thechapter}
\chead[]{}
\rhead[\thesection]{\thepage}

\lfoot[]{}
\cfoot[]{}
\rfoot[]{}

\renewcommand{\headrulewidth}{1.5pt}
\renewcommand{\footrulewidth}{0pt}


% 1ex - É a largura da letra x (minúscula)
% 1em - É a largura da letra M (Maiúscula)
\setlength{\parindent}{3ex}         % Tamanho do parágrafo
\setlength{\parskip}{0em}           % Espaço entre parágrafos

\makeatletter
\expandafter\def\expandafter\normalsize\expandafter{%
    \normalsize
    \setlength{\abovedisplayskip}{0em}
    \setlength{\belowdisplayskip}{0em}
    \setlength{\abovedisplayshortskip}{0em}
    \setlength{\belowdisplayshortskip}{0em}
}
\makeatother

% EOF

\RequirePackage{listings}

\RequirePackage{tikz, tikz-cd}      % Insercao de imagens configuraveis.

\usetikzlibrary{shapes.geometric, arrows}
\tikzstyle{startstop} = [rectangle, rounded corners, minimum width=3cm, minimum height=1cm,text centered, draw=black]
\tikzstyle{io} = [trapezium, trapezium left angle=70, trapezium right angle=110, minimum width=3cm, minimum height=1cm, text centered, draw=black]
\tikzstyle{process} = [rectangle, minimum width=3cm, minimum height=1cm, text centered, draw=black]
\tikzstyle{decision} = [diamond, minimum width=3cm, minimum height=1cm, text centered, draw=black]
\tikzstyle{conector} = [circle, minimum width=0.5cm, minimum height=0.5cm, text centered, draw=black]
\tikzstyle{arrow} = [thick,->,>=stealth]
\tikzstyle{line} = [draw, -latex']

% see https://latexcolor.com/
\definecolor{codegreen}{rgb}{0,0.6,0}
\definecolor{codegray}{rgb}{0.5,0.5,0.5}
\definecolor{amaranth}{rgb}{0.9, 0.17, 0.31}
\definecolor{theBlack}{gray}{0.0}
\definecolor{theWhite}{gray}{0.9}

\RequirePackage{caption}
\DeclareCaptionFont{white}{\color{white}}
\DeclareCaptionFormat{listing}{\hspace*{-0.4pt}\colorbox{gray}{\parbox{\textwidth}{#1#2#3}}}
\captionsetup[lstlisting]{format=listing,labelfont=white,textfont=white}

\renewcommand{\lstlistingname}{Código}

\lstset{
   backgroundcolor=\color{gray!5!white},   % choose the background color; you must add \usepackage{color} or \usepackage{xcolor}; should come as last argument
   basicstyle=\mdseries\ttfamily\footnotesize\color{black}, % the size of the fonts that are used for the code
   breakatwhitespace=false,         % sets if automatic breaks should only happen at whitespace
   breaklines=true,                 % sets automatic line breaking
   captionpos=t,                    % sets the caption-position to bottom
   columns=fixed,                   % Using fixed column width (for e.g. nice alignment)
   escapechar=µ,
   numbers=none,                    % Not use numbers
   keepspaces=true,                 % keeps spaces in text, useful for keeping indentation of code (possibly needs columns=flexible)
   showspaces=false,                % show spaces everywhere adding particular underscores; it overrides 'showstringspaces'
   showstringspaces=false,          % underline spaces within strings only
   showtabs=false,                  % show tabs within strings adding particular underscores
   tabsize=3, 	                     % sets default tabsize to 3 spaces
   %frame=single,	                  % adds a frame around the code
   %rulecolor=\color{black},         % if not set, the frame-color may be changed on line-breaks within not-black text (e.g. comments (green here))
}

\lstdefinestyle{c}{
   language=C, % the language of the code (can be overrided per snippet)
   commentstyle=\color{codegray}, % comment style
   keywordstyle={\color{codegreen}\bfseries},
   stringstyle=\color{amaranth} % string literal style
}

\lstdefinestyle{f90}{
   language=Fortran, % the language of the code (can be overrided per snippet)
   commentstyle=\color{codegray}, % comment style
   keywordstyle={\color{codegreen}\bfseries},
   stringstyle=\color{amaranth} % string literal style
}

\lstdefinestyle{java}{
   language=Java, % the language of the code (can be overrided per snippet)
   commentstyle=\color{codegray}, % comment style
   keywordstyle={\color{codegreen}\bfseries},
   stringstyle=\color{amaranth} % string literal style
}

\lstdefinestyle{py}{
   language=Python, % the language of the code (can be overrided per snippet)
   commentstyle=\color{codegray}, % comment style
   keywordstyle={\color{codegreen}\bfseries},
   stringstyle=\color{amaranth} % string literal style
}

\lstdefinestyle{gnuplot}{
   language=Gnuplot, % the language of the code (can be overrided per snippet)
   commentstyle=\color{codegray}, % comment style
   keywordstyle={\color{codegreen}\bfseries},
   stringstyle=\color{amaranth} % string literal style
}

\lstdefinestyle{bash}{
   language=bash, % the language of the code (can be overrided per snippet)
   commentstyle=\color{codegray}, % comment style
   keywordstyle={\color{codegreen}\bfseries},
   stringstyle=\color{amaranth}, % string literal style
   morekeywords={apt, pacman, yum, zypper, dnf, dns, mkdir, cp, configure, make, tar},
}

\lstdefinestyle{pseudo}{
   commentstyle=\color{codegray}, % comment style
   keywordstyle={\color{codegreen}\bfseries},
   stringstyle=\color{amaranth}, % string literal style
   morekeywords={inicio, fim, para, ate, subrotinas, se, senao, entao, funcao, inteiro, enquanto, gnuplot, logico, verdadeiro, subrotina, variavel, inicialize},
   morestring=[b]",
   morecomment={[l]//},
   inputencoding=utf8,
   extendedchars=true,
   literate=%
      {á}{{\'a}}1 {é}{{\'e}}1 {í}{{\'i}}1 {ó}{{\'o}}1 {ú}{{\'u}}1
      {Á}{{\'A}}1 {É}{{\'E}}1 {Í}{{\'I}}1 {Ó}{{\'O}}1 {Ú}{{\'U}}1
      {à}{{\`a}}1 {è}{{\`e}}1 {ì}{{\`i}}1 {ò}{{\`o}}1 {ù}{{\`u}}1
      {À}{{\`A}}1 {È}{{\'E}}1 {Ì}{{\`I}}1 {Ò}{{\`O}}1 {Ù}{{\`U}}1
      {ä}{{\"a}}1 {ë}{{\"e}}1 {ï}{{\"i}}1 {ö}{{\"o}}1 {ü}{{\"u}}1
      {Ä}{{\"A}}1 {Ë}{{\"E}}1 {Ï}{{\"I}}1 {Ö}{{\"O}}1 {Ü}{{\"U}}1
      {â}{{\^a}}1 {ê}{{\^e}}1 {î}{{\^i}}1 {ô}{{\^o}}1 {û}{{\^u}}1
      {Â}{{\^A}}1 {Ê}{{\^E}}1 {Î}{{\^I}}1 {Ô}{{\^O}}1 {Û}{{\^U}}1
      {ã}{{\~a}}1 {ẽ}{{\~e}}1 {ĩ}{{\~i}}1 {õ}{{\~o}}1 {ũ}{{\~u}}1
      {Ã}{{\~A}}1 {Ẽ}{{\~E}}1 {Ĩ}{{\~I}}1 {Õ}{{\~O}}1 {Ũ}{{\~U}}1
      {œ}{{\oe}}1 {Œ}{{\OE}}1 {æ}{{\ae}}1 {Æ}{{\AE}}1 {ß}{{\ss}}1
      {ű}{{\H{u}}}1 {Ű}{{\H{U}}}1 {ő}{{\H{o}}}1 {Ő}{{\H{O}}}1
      {ç}{{\c c}}1 {Ç}{{\c C}}1 {ø}{{\o}}1 {å}{{\r a}}1 {Å}{{\r A}}1
      {€}{{\euro}}1 {£}{{\pounds}}1 {«}{{\guillemotleft}}1
      {»}{{\guillemotright}}1 {ñ}{{\~n}}1 {Ñ}{{\~N}}1 {¿}{{?`}}1 {¡}{{!`}}1
      {←}{{$\leftarrow$}}1
      {!=}{{$\neq$}}1
}

\makeatletter
% \expandafter for the case that the filename is given in a command
\newcommand{\replunderscores}[1]{\expandafter\@repl@underscores#1_\relax}

\def\@repl@underscores#1_#2\relax{%
    \ifx \relax #2\relax
        % #2 is empty => finish
        #1%
    \else
        % #2 is not empty => underscore was contained, needs to be replaced
        #1%
        \textunderscore
        % continue replacing
        % #2 ends with an extra underscore so I don't need to add another one
        \@repl@underscores#2\relax
    \fi
}
\makeatother

\RequirePackage[most]{tcolorbox}
\RequirePackage{fontawesome5}       % Permite o uso da fonte Font Awesome

\newcommand{\inserircodigo}[2]{
   \begin{tcolorbox}[
      colback=gray!5!white,
      colframe=gray!75!black,
      title=\ttfamily\replunderscores{#2}
   ]
      \lstinputlisting[style = #1]{./SourceCodes/#2}
   \end{tcolorbox}
}


\usepackage{hyperref, bookmark}         % Insercao de hiperreferencias.
\usepackage{xcolor, color}              % hiperreferencias coloridas
\usepackage[hyphenbreaks]{breakurl}     % Quebra linha nno comando \url
\usepackage{url}                        % Bibliografia padrão ABNT.

% Por praticidade, em textos não academicos, prefiro usar como segue
%\RequirePackage[num, overcite]{abntex2cite} % Bibliografia padrão ABNT.
%\citebrackets[]                             % Citações com colchetes.

\hypersetup{
   pdfcreator={LaTeX with abnTeX2},
   bookmarksdepth=4,
   pagebackref=true,
   breaklinks=true,
   % Caso o documento seja impresso, marque a opção abaixo como false,
   % caso contrário, recomendo que matenha como true.
   colorlinks=true,             % false: boxed links; true: colored links
   linkcolor=blue(ryb),         % cor das ligações internas
   citecolor=darkmidnightblue,  % cor das chamadas bibliográficas
   urlcolor=teal,               % cor dos URL
   linkbordercolor=0.01 .28 1,  % cor das ligações internas
   citebordercolor=0 .2 .4,     % cor das chamadas bibliográficas
   urlbordercolor=0 .5 .5,      % cor dos URL
}
%  Configuração do PDF
\AtBeginDocument{
   \hypersetup{
      pdftitle={\imprimirtitulo}, 
      pdfauthor={\imprimirautor},
   }
}

%  Encoding: UTF-8
%
% %%%%%%%%%%%%%%%%%%%%%%%%%%%%%%%%%%%%%%%%%%%%%%%%%%%%%%%%%%%%%%%%%%%%%%%%%%%% %
%  Copyright (C) 2023 I. F. F. dos SANTOS
%
%  Permission is hereby granted, free of charge, to any person obtaining a copy 
%  of this software and associated documentation files (the “Software”), to 
%  deal in the Software without restriction, including without limitation the 
%  rights to use, copy, modify, merge, publish, distribute, sublicense, and/or
%  sell copies of the Software, and to permit persons to whom the Software is 
%  furnished to do so, subject to the following conditions:
%
%  The above copyright notice and this permission notice shall be included in 
%  all copies or substantial portions of the Software.
%
%  THE SOFTWARE IS PROVIDED “AS IS”, WITHOUT WARRANTY OF ANY KIND, EXPRESS OR 
%  IMPLIED, INCLUDING BUT NOT LIMITED TO THE WARRANTIES OF MERCHANTABILITY, 
%  FITNESS FOR A PARTICULAR PURPOSE AND NONINFRINGEMENT. IN NO EVENT SHALL THE 
%  AUTHORS OR COPYRIGHT HOLDERS BE LIABLE FOR ANY CLAIM, DAMAGES OR OTHER 
%  LIABILITY, WHETHER IN AN ACTION OF CONTRACT, TORT OR OTHERWISE, ARISING 
%  FROM, OUT OF OR IN CONNECTION WITH THE SOFTWARE OR THE USE OR OTHER DEALINGS 
%  IN THE SOFTWARE.
% %%%%%%%%%%%%%%%%%%%%%%%%%%%%%%%%%%%%%%%%%%%%%%%%%%%%%%%%%%%%%%%%%%%%%%%%%%%% %

\usepackage{lipsum}    %
\usepackage{lettrine}  % Letra capitular
\usepackage{enumitem}  %
\setlist[enumerate]{nosep, topsep=-\parskip}
\setlist[itemize]{nosep, topsep=-\parskip}

\usepackage{graphicx}                  % Inclusão de figuras.
\usepackage{float}                     % Para posicionar figuras corretamente.
\usepackage{subcaption}                % Para legendas nas subfiguras.
\usepackage[cleanup]{gnuplottex}       % Para uso do GNUPlotTex.
\usepackage[tickmarkheight=5pt]{todonotes}                 % Para TODO

% EOF



% --------------------------------------
% Informações de dados para CAPA e FOLHA DE ROSTO
% --------------------------------------
% Nesta parte ainda não são permitidos acentos, segue um dicionário
% cedilha ç = \c{c}, Ç = \c{C}
% agudo á = \'{a}, Á = \'{A}
% agudo é = \'{e}, É = \'{E}
% agudo í = \'{i}, Í = \'{I}
% agudo ó = \'{o}, Ó = \'{O}
% agudo ú = \'{u}, Ú = \'{U}
\titulo{%
   The Tuim Project \\ \small{Version 0.2}%
}
\autor{%
   I. F. F. dos{ }SANTOS\thanks{\href{mailto:ismaellxd@gmail.com}{ismaellxd@gmail.com}}%
}
\local{Maceió}

% informações do PDF
\hypersetup{
   english,
   pdftitle={\imprimirtitulo}, 
   pdfauthor={\imprimirautor},
   pdfcreator={LaTeX with abnTeX2},
   %pagebackref=true,
   bookmarksdepth=4,
   colorlinks=true,            % false: boxed links; true: colored links
   linkcolor=blue,             % color of internal links
   citecolor=blue,             % color of links to bibliography
   filecolor=mageta,          % color of file links
   urlcolor=blue
}
\makeindex

\begin{document}
\selectlanguage{english}
% ------------------------------------------------------------------------------
% ELEMENTOS PRÉ-TEXTUAIS
% ------------------------------------------------------------------------------
\imprimircapa

\newpage
\tableofcontents
\newpage
% ------------------------------------------------------------------------------
% ELEMENTOS TEXTUAIS
% ------------------------------------------------------------------------------
\textual

\chapter{Introduction}

This document describe the usage of the Tuim's ELF interpreter
and the Application Binary Interface (ABI) that conforming programs and libraries
must follow in order to interoperate.

\section{How it works}

Supose a executable \texttt{ola.elf} that want to print on the terminal
the message
"Ola mundo!", it request such service (write some bytes on terminal)
to the execution environment by calling the function \texttt{puts}.
The executable \texttt{ola.elf} do not know how \texttt{puts} is implemented but
it know that it's implemented in the library \texttt{libc.so}.
If the execution environment have a \texttt{libc.so} library then the function
\texttt{puts} can be sucessfully called by the executable and the message
"Ola mundo!" can be written to the terminal.

Since there is no Operational System (OS) dependent code in the executable
\texttt{ola.elf} then \textbf{any} OS that understand instructions in such executable
can execute it natively, without the need of recompilation or virtualization.
Of course OS dependent instruction can be found in the library \texttt{libc.so},
therefore any OS need to have its own compilation of \texttt{libc.so}.

This is how a executable can be "portable" in the sence that more than one
OS can execute it. Some libraries also can be "portable" but a lot of libraries
need to have OS dependent instructions.

\section{Object files}

The Executable and Linking Format (ELF) is the object file format used
by the Tuim project, specification for those files can be found in
chapters 4 and 5 of System V ABI\cite{sysv} base document.
Tuim only use files with \texttt{EI\_OSABI} field of the ELF header equals
to \texttt{ELFOSABI\_NONE},
this means that all operating system-specific semantics
allowed by the document are explicitily ignored,
however processor-specific semantics are handled by the Tuim project.

\section{Interpreter}

The ELF Interpreter used by the Tuim project is the software that
parse ELF files in order to load its instructions on computer's memory.
The Interpreter only can load (and relocate) Position Independent Code (PIC),
as a consequece all virtual addresses on both executables and shared objects
are treated as offsets from the base address.

\chapter{Execution Environment}

The Tuim's \texttt{run} command is the command used to pass a program
to the Interpreter, then the Interpreter prepare all for execution and
the \texttt{run} command start the execution of the program.

The correctness of \texttt{run} and Interpreter's
work is constrained to the correct
environment configuration.
The environment variable \texttt{TUIM\_HOME}
shall holds the directory path where Tuim is installed.
When Tuim is used to run a program named \texttt{prog} then it find
for the executable placed at \texttt{bin/prog.elf} inside \texttt{TUIM\_HOME}.
If the machine that must run the program is the host machine
then the Interpreter find for libraries inside the subdirectory
\texttt{lib} of \texttt{TUIM\_HOME},
otherwise, if is a machine named \texttt{m} that must run the program then
the Tuim first find for another implementation of the Interpreter
in the library \texttt{lib/libtuim-m.so} and that Interpreter find for libraries
inside \textt{lib/m} subdirectory.

\section{Executing from command line}

\subsubsection*{SYNOPSIS}

\begin{lstlisting}[style=bash]
tuim run [-m machine] <program|executable> [arguments...]
\end{lstlisting}

\subsubsection*{DESCRIPTION}

Execute the Tuim's \texttt{run} command,
optionally using \texttt{machine} as machine,
using \texttt{program} as program's name,
if such name contains slashes
it is interpreted as the a \texttt{executable} path.

\section{Executing from C code}

\subsubsection*{SYNOPSIS}

\begin{lstlisting}[style=c]
#include <tuim.h>

int tuim_run(const char *machine, char **argv);
\end{lstlisting}

\subsubsection*{DESCRIPTION}

The function \texttt{tuim\_run} execute the Tuim's \texttt{run} command.
If the pointer \texttt{machine} is non-null it
uses the null-terminated string pointed to by
\texttt{machine} as machine's name.
It uses the null-terminated string pointed to by
\texttt{argv[0]} as program's name,
if such string contains slashes
it is interpreted as the a executable path.

\chapter{Development Environment}

In a development environment not only the Tuim's CLI is installed but a
lot of libraries and utilities shall be available.

The Tuim project aims to allow interoperability among languages,
this is possible for languages where the mapping
betwen the language and machine code/ELF is well defined.
In general processor suplements for the System V ABI design the mapping
from C to machine code/ELF and some languages design the mapping
from the language to C.
Therefore the \textit{natural} languages used to develop applications
for Tuim are Assembly, C and languages that interfaces with C,
however the author encourage programing language developers to desing
the mapping to machine code/ELF, thus increasing interoperability.
A complete list of officially supported languages is placed bellow in
build rules.

% -----------------
% SECTION
% -----------------
\section{The source tree specification file}

The source tree specification file is a file named
\texttt{build.yaml} in the source directory of a project.
This file is used by the \texttt{tuim build} command to build a object code
from source code and use the YAML\cite{yaml} format.

Each document in \texttt{build.yaml} describe a object code
and list all the source files used to build it,
file path shall be relative to the \texttt{build.yaml}.
For example, the following file is used in the test project \texttt{ola\_mundo}.

\begin{lstlisting}[style=yaml]
executable: ola.elf
entry: main
dependencies:
  1: libc.so
src:
  - ola.c
\end{lstlisting}

When the command \texttt{tuim build src} is executed the procedure look like
the following shell code.

\begin{lstlisting}[style=bash]
mkdir tmp
cc -o tmp/ola.o src/ola.c
mkdir bin
ld -o bin/ola.elf -e main -l c tmp/ola.o
\end{lstlisting}

Executables are always placed at a \texttt{bin} directory and
shared objects are always placed at a \texttt{lib} directory,
note that a document can contain either a \texttt{executable} or
a \texttt{library} field but \textbf{not} both,
also executables need a \texttt{entry} field with the entry point symbol
and libraries can have \texttt{init} and \texttt{fini} fields with the symbol
of a initialization and a termination function.
A \texttt{tmp} directory is always created to place relocatable objects
and intermediate representation objects among other files.
Dependencies are linked in the specified order.

\subsection{Build rules}

The tools described in this chapter are used to specify the
build rules for each file extension:

\begin{itemize}

\item GNU style Assembly \texttt{.asm}
\begin{lstlisting}[basicstyle=\tiny,style=bash]
as -o tmp/$(dirname ${file})/$(basename ${file} .asm).o src/${file}
\end{lstlisting}

\item GNU style preprocessed Assembly \texttt{.S}
\begin{lstlisting}[basicstyle=\tiny,style=bash]
cpp -o tmp/$(dirname ${file})/$(basename ${file} .S).asm src/${file}
as -o tmp/$(dirname ${file})/$(basename ${file} .S).o tmp/$(basename ${file} .S).asm
\end{lstlisting}

\item Freestanding C23 \texttt{.c}
\begin{lstlisting}[basicstyle=\tiny,style=bash]
cc -o tmp/$(dirname ${file})/$(basename ${file} .c).o src/${file}
\end{lstlisting}

\end{itemize}

Note that if the \texttt{-D} option is used in the \texttt{tuim build} call
then all file extensions that need to be
preprocessed inherints this options in the previous rules.

% -----------------
% SECTION
% -----------------
\section{Command line utilities}

\newpage
% -----------------
% SUBSECTION
% -----------------
\subsection{Linker}

\subsubsection*{NAME}

\texttt{ld} -- The linker

\subsubsection*{SYNOPSIS}

\begin{lstlisting}[style=bash]
ld [OPTIONS] objfile ...
\end{lstlisting}

\subsubsection*{DESCRIPTION}

\todo[inline]{Writte about this}

\subsubsection*{OPTIONS}

\begin{description}[style=multiline,leftmargin=5cm]
   \item[\texttt{{-}{-}shared}]
   Enable generation of shared object.
   \item[\texttt{-e <symbol>}]
   Enable generation of executable and
   define \texttt{<symbol>} as the entry point.
   \item[\texttt{-o <name>}]
   Define the output file's name,
   if not present defaults to the same input name but with
   extension changed to \texttt{elf} (for executables)
   or \texttt{so} (for shared objects).
   \item[\texttt{-l <library>}]
   Add a \texttt{DT\_NEEDED} entry on the dynamic table for the
   \texttt{lib<library>.so} dependency.
   \todo[fancyline]{Complete this list.}
\end{description}

\subsubsection*{ENVIRONMENT VARIABLES}

\begin{description}[style=multiline,leftmargin5cm]
   \item[\texttt{\detokenize{LD_LIBRARY_PATH}}]
   \ldots\todo[fancyline]{Fiz this}
\end{description}

\newpage
% -----------------
% SUBSECTION
% -----------------
\subsection{Assembler}

\texttt{as} - The assembler

\subsubsection*{SYNOPSIS}

\begin{lstlisting}[style=bash]
as [OPTIONS] asmfile ...
\end{lstlisting}

\subsubsection*{DESCRIPTION}

\todo[inline]{Writte about this}

\subsubsection*{OPTIONS}

\begin{description}[style=multiline,leftmargin=5cm]
   \item[\texttt{-o <name>}]
   Define the output file's name,
   if not present defaults to the same input name but with
   extension changed to \texttt{o}.
\end{description}

\newpage
% -----------------
% SUBSECTION
% -----------------
\subsection{C Compiler}

\subsubsection*{NAME}

\begin{lstlisting}[style=bash]
cc - The C compiler
\end{lstlisting}

\noindent SYNOPSIS

\begin{lstlisting}[style=bash]
cc [OPTIONS] srcfile ...
\end{lstlisting}

\subsubsection*{DESCRIPTION}

The C compiler is a software to compile source code writter in the
C Programing Language,
it outputs relocatable files.
\todo[fancyline]{Improve this}

\subsubsection*{OPTIONS}

\begin{description}[style=multiline,leftmargin=5cm]
   \item[\texttt{-o <name>}]
   Define the output file's name,
   if not present defaults to the same input name but with
   extension changed to \texttt{o}(for relocatable objects).
   \item[\texttt{-I <path>}]
   Define the \texttt{<path>} for find header files.
   \todo[fancyline]{Complete this list}
\end{description}

% EOF

\chapter{Standard libraries}
\label{cap:libraries}

% -----------------
% SECTION
% -----------------
\section{Runtime libraries}

A program may request libraries as shared objects,
if the shared object name has no slash then the loader find for it
first on directories specified by the environment variable
\texttt{LD\_LIBRARY\_PATH} and then in \texttt{<TUIM\_HOME>/lib}.

All shared object names are reserved by this ABI,
if a  need a shared object for porposes others than
specified in this document then it need provide the shared object and
the environment variable \texttt{LD\_LIBRARY\_PATH} need to be set
after program loading.

All global and weak symbols prefixed with \texttt{\_\_tuim\_}
are reserved by this ABI,
a conforming program shall not provide definitions for these symbols,
even if it not request any library.



% ------------------------------------------------------------------------------
% ELEMENTOS PÓS-TEXTUAIS
% ------------------------------------------------------------------------------
   \postextual
   % ---------------------------------------------------------------------------
   % Referências bibliográficas
   % ---------------------------------------------------------------------------
   \bibliography{references}
   \apendices
   \renewcommand{\thesection}{\Alph{section}}

\section{Targets}
\label{apendice:target}

Allowed targets are
\begin{itemize}
   \item \texttt{x86}
   \item \texttt{x86\_64}
   \item \texttt{arm}
   \item \texttt{aarch64}
   \item \texttt{riscv32}
   \item \texttt{riscv64}
\end{itemize}

\end{document}
