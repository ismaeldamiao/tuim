\chapter{Development Environment}

In order to make easy build projects to run using Tuim,
this chapter describe command utilities in order to build
libraries and programs for Tuim,
however and obviously developers are allowed to use any others tools.

Developers using tools on this chapter can use
a POSIX compliant shell\cite{posix}.
Shell scripts are used to build libraries and programs,
the basic step to build a project using conventions ans tools in this chapter is
the execution of the following commands on the project's source directory.

\begin{lstlisting}[style=bash]
. "${TUIM_HOME}/share/dev-host.sh"
sh scripts/build.sh
\end{lstlisting}

The source file \texttt{\$\{TUIM\_HOME\}/share/dev-host.sh}
set up environment variables and aliases that define tools in this chapter
and is suitable to build code for the host machine.
To build code for another machine, source files
\texttt{\$\{TUIM\_HOME\}/share/dev-<arch>.sh} shall be used instead,
where \texttt{<arch>} is one Instruction Set Architecture (ISA)
supported by Tuim.
On Tuim's source directory \texttt{skel} such files are implemented as
wrappers to LLVM.

\newpage
% -----------------
% SECTION
% -----------------
\section{Linker}

\subsubsection*{NAME}

\texttt{ld} - The linker

\subsubsection*{SYNOPSIS}

\begin{lstlisting}[style=bash]
. "${TUIM_HOME}/share/dev-host.sh"

ld [OPTIONS] objfile ...
\end{lstlisting}

\subsubsection*{DESCRIPTION}

...

\begin{lstlisting}[style=bash]
alias ld="${LD} ${LDFLAGS}"
\end{lstlisting}

\subsubsection*{OPTIONS}

\begin{description}[style=multiline,leftmargin=5cm]
   \item[\texttt{{-}{-}shared}]
   Enable generation of shared object.
   \item[\texttt{-e <symbol>}]
   Enable generation of executable and
   define \texttt{<symbol>} as the entry point.
   \item[\texttt{-o <name>}]
   Define the output file's name,
   if not present defaults to the same input name but with
   extension changed to \texttt{elf} (for executables)
   or \texttt{so} (for shared objects).
   \item[\texttt{-l <library>}]
   Add a \texttt{DT\_NEEDED} entry on the dynamic table for the
   \texttt{lib<library>.so} dependency.
\end{description}

\subsubsection*{ENVIRONMENT VARIABLES}

\begin{description}[style=multiline,leftmargin5cm]
   \item[{\texttt{LD\_LIBRARY\_PATH}}]
   ...
\end{description}

\newpage
% -----------------
% SECTION
% -----------------
\section{Assembler}

\texttt{as} - The assembler

\subsubsection*{SYNOPSIS}

\begin{lstlisting}[style=bash]
. "${TUIM_HOME}/share/dev-host.sh"

as [OPTIONS] asmfile ...
\end{lstlisting}

\subsubsection*{DESCRIPTION}

...

\begin{lstlisting}[style=bash]
alias as="${ASM} ${ASMFLAGS}"
\end{lstlisting}

\subsubsection*{OPTIONS}

\begin{description}[style=multiline,leftmargin=5cm]
   \item[\texttt{-o <name>}]
   Define the output file's name,
   if not present defaults to the same input name but with
   extension changed to \texttt{o}.
\end{description}

\newpage
% -----------------
% SECTION
% -----------------
\section{C Compiler}

\subsubsection*{NAME}

\begin{lstlisting}[style=bash]
cc - The C compiler
\end{lstlisting}

\noindent SYNOPSIS

\begin{lstlisting}[style=bash]
cc [OPTIONS] srcfile ...
\end{lstlisting}

\subsubsection*{DESCRIPTION}

The C compiler is a software to compile source code writter in the
C Programing Language,
it outputs relocatable files.

\begin{lstlisting}[style=bash]
alias cc="${CC} ${CFLAGS}"
\end{lstlisting}

\subsubsection*{OPTIONS}

\begin{description}[style=multiline,leftmargin=5cm]
   \item[\texttt{-o <name>}]
   Define the output file's name,
   if not present defaults to the same input name but with
   extension changed to \texttt{o}(for relocatable objects).
   \item[\texttt{-I <path>}]
   Define the \texttt{<path>} for find header files.
\end{description}
