\chapter{Development Environment}

In a development environment not only the Tuim's CLI is installed but a
lot of libraries and utilities shall be available.

The Tuim project aims to allow interoperability among languages,
this is possible for languages where the mapping
betwen the language and machine code/ELF is well defined.
In general processor suplements for the System V ABI design the mapping
from C to machine code/ELF and some languages design the mapping
from the language to C.
Therefore the \textit{natural} languages used to develop applications
for Tuim are Assembly, C and languages that interfaces with C,
however the author encourage programing language developers to desing
the mapping to machine code/ELF, thus increasing interoperability.
A complete list of officially supported languages is placed bellow in
build rules.

% -----------------
% SECTION
% -----------------
\section{The source tree specification file}

The source tree specification file is a file named
\texttt{build.yaml} in the source directory of a project.
This file is used by the \texttt{tuim build} command to build a object code
from source code and use the YAML\cite{yaml} format.

Each document in \texttt{build.yaml} describe a object code
and list all the source files used to build it,
file path shall be relative to the \texttt{build.yaml}.
For example, the following file is used in the test project \texttt{ola\_mundo}.

\begin{lstlisting}[style=yaml]
executable: ola.elf
entry: main
dependencies:
  1: libc.so
src:
  - ola.c
\end{lstlisting}

When the command \texttt{tuim build src} is executed the procedure look like
the following shell code.

\begin{lstlisting}[style=bash]
mkdir tmp
cc -o tmp/ola.o src/ola.c
mkdir bin
ld -o bin/ola.elf -e main -l c tmp/ola.o
\end{lstlisting}

Executables are always placed at a \texttt{bin} directory and
shared objects are always placed at a \texttt{lib} directory,
note that a document can contain either a \texttt{executable} or
a \texttt{library} field but \textbf{not} both,
also executables need a \texttt{entry} field with the entry point symbol
and libraries can have \texttt{init} and \texttt{fini} fields with the symbol
of a initialization and a termination function.
A \texttt{tmp} directory is always created to place relocatable objects
and intermediate representation objects among other files.
Dependencies are linked in the specified order.

\subsection{Build rules}

The tools described in this chapter are used to specify the
build rules for each file extension:

\begin{itemize}

\item GNU style Assembly \texttt{.asm}
\begin{lstlisting}[basicstyle=\tiny,style=bash]
as -o tmp/$(dirname ${file})/$(basename ${file} .asm).o src/${file}
\end{lstlisting}

\item GNU style preprocessed Assembly \texttt{.S}
\begin{lstlisting}[basicstyle=\tiny,style=bash]
cpp -o tmp/$(dirname ${file})/$(basename ${file} .S).asm src/${file}
as -o tmp/$(dirname ${file})/$(basename ${file} .S).o tmp/$(basename ${file} .S).asm
\end{lstlisting}

\item Freestanding C23 \texttt{.c}
\begin{lstlisting}[basicstyle=\tiny,style=bash]
cc -o tmp/$(dirname ${file})/$(basename ${file} .c).o src/${file}
\end{lstlisting}

\end{itemize}

Note that if the \texttt{-D} option is used in the \texttt{tuim build} call
then all file extensions that need to be
preprocessed inherints this options in the previous rules.

% -----------------
% SECTION
% -----------------
\section{Command line utilities}

\newpage
% -----------------
% SUBSECTION
% -----------------
\subsection{Linker}

\subsubsection*{NAME}

\texttt{ld} -- The linker

\subsubsection*{SYNOPSIS}

\begin{lstlisting}[style=bash]
ld [OPTIONS] objfile ...
\end{lstlisting}

\subsubsection*{DESCRIPTION}

\todo[inline]{Writte about this}

\subsubsection*{OPTIONS}

\begin{description}[style=multiline,leftmargin=5cm]
   \item[\texttt{{-}{-}shared}]
   Enable generation of shared object.
   \item[\texttt{-e <symbol>}]
   Enable generation of executable and
   define \texttt{<symbol>} as the entry point.
   \item[\texttt{-o <name>}]
   Define the output file's name,
   if not present defaults to the same input name but with
   extension changed to \texttt{elf} (for executables)
   or \texttt{so} (for shared objects).
   \item[\texttt{-l <library>}]
   Add a \texttt{DT\_NEEDED} entry on the dynamic table for the
   \texttt{lib<library>.so} dependency.
   \todo[fancyline]{Complete this list.}
\end{description}

\subsubsection*{ENVIRONMENT VARIABLES}

\begin{description}[style=multiline,leftmargin5cm]
   \item[\texttt{\detokenize{LD_LIBRARY_PATH}}]
   \ldots\todo[fancyline]{Fiz this}
\end{description}

\newpage
% -----------------
% SUBSECTION
% -----------------
\subsection{Assembler}

\texttt{as} - The assembler

\subsubsection*{SYNOPSIS}

\begin{lstlisting}[style=bash]
as [OPTIONS] asmfile ...
\end{lstlisting}

\subsubsection*{DESCRIPTION}

\todo[inline]{Writte about this}

\subsubsection*{OPTIONS}

\begin{description}[style=multiline,leftmargin=5cm]
   \item[\texttt{-o <name>}]
   Define the output file's name,
   if not present defaults to the same input name but with
   extension changed to \texttt{o}.
\end{description}

\newpage
% -----------------
% SUBSECTION
% -----------------
\subsection{C Compiler}

\subsubsection*{NAME}

\begin{lstlisting}[style=bash]
cc - The C compiler
\end{lstlisting}

\noindent SYNOPSIS

\begin{lstlisting}[style=bash]
cc [OPTIONS] srcfile ...
\end{lstlisting}

\subsubsection*{DESCRIPTION}

The C compiler is a software to compile source code writter in the
C Programing Language,
it outputs relocatable files.
\todo[fancyline]{Improve this}

\subsubsection*{OPTIONS}

\begin{description}[style=multiline,leftmargin=5cm]
   \item[\texttt{-o <name>}]
   Define the output file's name,
   if not present defaults to the same input name but with
   extension changed to \texttt{o}(for relocatable objects).
   \item[\texttt{-I <path>}]
   Define the \texttt{<path>} for find header files.
   \todo[fancyline]{Complete this list}
\end{description}

% EOF
