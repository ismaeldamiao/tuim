\chapter{Execution Environment}

The execution environment is used to start the execution of a application.
When a application is requested to be executed it
may request some service from libraries.
The Tuim's ELF Interpreter is the component responsible for
find and load the application and all libraries it request.

Both application and libraries are stored in object codes,
the format used is the Executable and Linking Format (ELF)
-- see chapters 4 and 5 of System V ABI\cite{sysv} base document --,
applications are stored in executables and libraries in shared objects.
The Interpreter only are granted to sucessful load
ELF files without non standard features, that is,
operating system-specific semantics are ignored
-- of course, processor-specific semantics are handled by the Interpreter.

% --------
% SECTION
% --------
\section{Interfaces for the Interpreter}

% SUBSECTION
\subsection{C API}

The only official API is written for the C Programin Language\cite{ISO9899}.
Identifier implemented in the library are exposed in the
header \texttt{tuim.h}.

The macros in the header \texttt{tuim.h} are:

\begin{itemize}
   \item \texttt{TUIM\_VERSION\_MAJOR}
   \item \texttt{TUIM\_VERSION\_MINOR}
   \item \texttt{TUIM\_HOME}
   \item \texttt{LD\_LIBRARY\_PATH}
\end{itemize}

\subsubsection{The \texttt{tuim\_setenv} function}

\subsubsection*{SYNOPSIS}

\begin{lstlisting}[style=c]
#include <tuim.h>

int tuim_setenv(int env, char *value);
\end{lstlisting}

The \texttt{tuim\_setenv} function set to \texttt{value} a internal variable
used by the Interpreter. \texttt{env} shall be any of \texttt{TUIM\_HOME}
or \texttt{LD\_LIBRARY\_PATH} macros.

\subsubsection{The \texttt{tuim\_run} function}

\subsubsection*{SYNOPSIS}

\begin{lstlisting}[style=c]
#include <tuim.h>

int tuim_run(const char *machine, const char *exec_path, char **argv);
\end{lstlisting}

\subsubsection*{DESCRIPTION}

The \texttt{tuim\_run} function
execute the application where executable path is pointed to by
\texttt{exec\_path}, the application entry point is treated like if
it is the address of a function with the signature, in C, given by

\begin{lstlisting}[style=c]
int main(int argc, char **argv);
\end{lstlisting}

Where \texttt{argc} is computed to be such that \texttt{argv[argc] == NULL}.

If \texttt{machine} is \texttt{NULL}
the Interperter find for libraries using environment variables,
it find first on paths indicated by \texttt{LD\_LIBRARY\_PATH}
and then in \texttt{<TUIM\_HOME>/lib}.

If \texttt{machine} is not \texttt{NULL} the \texttt{tuim\_run} function
find for a alternative implementation of itself in
\texttt{<TUIM\_HOME>/opt/<machine>/tuim.so} shared object
and the Interpreter find for libraries in
\texttt{<TUIM\_HOME>/opt/<machine>/lib} directory.
This can be used, for example,
to execute the application in a virtual machine or in a external device.

% SUBSECTION
\subsection{CLI}

The CLI is suitable for systems on that one can
use a command line environment to request services.
The reference implementation are made in C using the API previous described.

\subsubsection{The \texttt{tuim} utility}

\subsubsection*{NAME}

\noindent
\texttt{tuim} -- Create, manage and run applications and libraries.

\subsubsection*{SYNOPSIS}

\begin{lstlisting}[style=bash]
tuim <command ...>
\end{lstlisting}

\subsubsection*{DESCRIPTION}

The \texttt{tuim} utility is hight-level command-line interface.
Commands are used with distincts purposes. The current available commands are:

\begin{itemize}

   \item \texttt{run [-m machine] <application|executable> [arguments...]}

   The \texttt{run} command execute the \texttt{application}
   or the application from the \texttt{executable},
   if the application name contain one or more slashes
   then it is interpreted as the executable path,
   otherwise the executable path in supposed to be
   \texttt{<TUIM\_HOME>/bin/application.elf}.
   A machine can be used to choose the correct implementation of the
   Interpreter for execute the application.
   The utility behave like calling the C API as

\begin{lstlisting}[style=c]
// tuim run lua script.lua
char *argv[] = { "lua", "script.lua", NULL };
tuim_run(NULL, "<TUIM_HOME>/bin/lua.elf", argv);
// tuim run -m riscv ./benchmark.elf
char *argv[] = { "./benchmark.elf", NULL };
tuim_run("riscv", "./benchmark.elf", argv);
// tuim run -m x86_64 game
char *argv[] = { "game", NULL };
tuim_run("x86_64", "<TUIM_HOME>/opt/x86_64/bin/game.elf", argv);
\end{lstlisting}

   \item \texttt{build [--target arch] [-D name[=definition]] <directory>}

   The \texttt{build} command build object code from source code
   using the \texttt{<directory>/build.yaml} source tree specification.
   The \texttt{-D} option is suitable for the control of languages that uses
   the C preprocessor.
   If the \texttt{--target} option is not used the command build object code
   as best it can for the host architecture, otherwise it build object code
   for the specified architecture.

\end{itemize}

% --------
% SECTION
% --------
\section{Standard libraries}
\label{section:libraries}

% SUBSECTION
\subsection{Compiler runtime library}

\subsubsection*{NAME}

\noindent
\texttt{libcrt} -- Compiler Runtime Library

\subsubsection*{SYNOPSIS}

\noindent
\texttt{libcrt.so}, \texttt{-l crt}

\subsubsection*{DESCRIPTION}

The \texttt{libcrt.so} shared object
expose symbols from the Compiler Runtime Library,
usually helper functions that the compiler need.
Symbols exposed by this shared object
begins with two underscores or an underscore followed by a capital letter.

Some processor suplements provide a specification for the
Compiler Runtime Library, in that case this is the library in the shared
object \texttt{libcrt.so}.
If the processor suplement do not provide such specification than
is a undefined bahavior the explicit use of
any symbol exposed by \texttt{libcrt.so} in source code,
these symbols are reserved for the use of the compiler.

\todo[inline]{%
   TODO: Choose a better design for this library.
}

% SUBSECTION
\subsection{The C standard library}

\subsubsection*{NAME}

\noindent
\texttt{libc} -- C Standard Library

\subsubsection*{SYNOPSIS}

\noindent
\texttt{libc.so}, \texttt{-l c}

\subsubsection*{DESCRIPTION}

The \texttt{libc.so} shared object
expose symbols from the C Standard Library, the latest version\cite{ISO9899}.
Symbols exposed by this shared object are symbols reserved by the C standard.

Several global and weak symbols
are reserved by the C standard,
a program requesting \texttt{libc.so}
shall not provide definitions for these symbols.
These are\footnote{%
   Note that the nomeclature used by the C standard is
   "identifier" instead of "symbol" as used by the ELF standard,
   however all external identifiers are mapped by the compiler in
   global or weak symbols.
   For more details please check the C standard\cite{ISO9899}.
}:

\begin{itemize}
   \item Symbols prefixed with an underscore (\texttt{\_}).
   \item Symbols defined by the standard library (in hosted environment).
   \item Symbols declared as reserved for the implementation
   or future use by the standard library.
   \item Symbols declared as potentially reserved.
\end{itemize}

