\section{Development Environment}

\lettrine{C}{ode}
produced withing The Tuim Project follows convections of reference
\citeonline{std_dev}
to be built.
In order to follow that convenctions a terminal using a POSIX compliant
shell is needed, usually Unix-like systems comes with compliant shells,
on windows there is a lot of option to get it,
one is to use the package \texttt{busybox}, it can be installed as follows:

\begin{lstlisting}[style=sh]
choco install busybox
# then start with
busybox sh
\end{lstlisting}

\dots

The script \texttt{tuim/script/development.sh}
shall be sourced\dots

\begin{lstlisting}[style=sh]
. "tuim/scripts/development.sh"

export TUIM_HOME="/opt/tuim"

cd tuim
make TUIM_HOME="${TUIM_HOME}" install
cd ..

. "${TUIM_HOME}/etc/profile.d/development.sh"

cd libkernel
make PREFIX="${TUIM_HOME}" install
cd ..

cd libc
make PREFIX="${TUIM_HOME}" install
cd ..

cd sh
make PREFIX="${TUIM_HOME}" install
cd ..
\end{lstlisting}

At the end is expected to have \texttt{\$\{TUIM\_HOME\}} as:

\begin{lstlisting}
/opt/tuim
|-- bin
|   `-- sh
|-- etc
|   |-- profile
|   `-- profile.d
|       `-- development.sh
`-- lib
    |-- libc.so
    `-- libkernel.so
\end{lstlisting}
