\section{Make Scripts}

Make Scripts are planed to be easy to use,
such that people that wants to build a project can easily do it.
However Make Scripts may be not so easy to implement the first time,
but developers are encouraged to write their Make Scripts
using modular programing, with reutilizable portions of code that can be easily
ported to others projects.

A Make Script is a shell script named \texttt{Make.sh}.
The following function shall be available on the terminal section
such in order to facilitate the usage of Make Scripts.

\begin{lstlisting}[style=sh]
make ( ) {
   [ -r "Make.sh" ] && { sh Make.sh $@; return $?; }
   echo "ERROR: No Make Script found."
   return 1
}
\end{lstlisting}

That shall available as a function, that way someone interested to use
POSIX Make may disable it.

\begin{lstlisting}[style=sh]
unset make
\end{lstlisting}

\subsection{Usage}

\begin{lstlisting}[style=sh]
make [macro=value...] [command]
\end{lstlisting}
